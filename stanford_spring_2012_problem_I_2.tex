\documentclass{amsart}
\usepackage{graphicx}
\graphicspath{{./}}
\usepackage{hyperref}
\usepackage{csvsimple}
\usepackage{longtable}
\usepackage{lscape}
\usepackage{epigraph}
\title{Stanford Spring 2012 Analysis Problem I.2}
\author{Zulfikar Moinuddin Ahmed}
\date{\today}
\begin{document}
\maketitle

\section{Disclaimer}

I am fascinated now by Barry Simon's account of the history of distribution theory in p. 511-514 of {\em Real Analysis} from his comprehensive Caltech graduate text. I am extremely pleased that Barry Simon has been with the human race.  He's a gift for people like me.

Well I will do this problem at some point, but I am 49, as you know, and I am much more interested in the history, just so interesting, and so difficult to find a way to learn about it too.  


\section{Barry Simon's Account}

The notion of "generalised functions", function-like objects that can be manipulated like functions has roots over 50 years earlier than Schwartz's formal definition of distribution in the late 1940s.  The earliest work came from electrical engineering and physics--most notably Heaviside's use of "operational calculus" and Dirac's use of his "delta-function" in formulation of quantum mechanics.

Oliver Heaviside (1850-1925) discussed formal derivative of the step function so he had a delta-function.  Paul A. M. Dirac (1902-84) used objects like $\delta( x_0^2  - x_1^2 - x_2^2 - x_3^2)$ which is delta function of a light cone.  Despite opposition Heaviside's and Dirac's formalisms became common among engineers and physicists.

In mathematics, Jacques Hadamard was trying to define the fundamental solution of the wave equation in his book of 1923.

A different slice came from Friedrichs, Bochner, and Sobolev who attempted in 1930s to define weak solutions of PDEs, what we would call the distributional solutions.  

Sobolev, in two papers of 1930s came closest to distribution theory.  For each $m$ he considered duals of $C^m_0(\mathbf{R}^{\nu})$.  J. Lutzen has a book, {\em The Prehistory of the Theory of Distributions}, Springer-Verlag, 1982.

Laurent Schwartz took the leap with duals of $C^{\infty}_0$ and Henri Cartan warned him that these things were 'too monstrous' according to Francois Treves.

So Laurent Schwartz took the recently developed theory of duality in locally convex spaces and found a general framework for development of distribution theory.  In 1945-46 he he presented the first steps to ordinary distribution theory and in 1948 he invented $\mathcal{S}(\mathbf{R}^n)$ and its dual to discuss Fourier transforms.  He codified this in a two-volume book on Theory of Distributions in 1951.  He won the Fields Medal in 1950.  

Quite a beautiful moment, in the history of the intellectual history of the human race, the birth of the Schwartz space and distributions.  

\section{Zulf's Thoughts}

For me 1950 is so long before I was born that it is not possible to fully understand the world from that time.  But I am quite impressed by this moment of examination of distribution theory.  Now I want to make some comments about where this ought to go.  I think that there is a huge gap between the theory of distributions and mathematical theory of wavelet analysis that requires a unity.

I think there is far more {\em algorithmic} understanding that is missing from the world for distribution theory, so that further developments occur in use of a {\em merging} of distribution theory and wavelet analysis in computational analysis with more sophisticated abilities for true leveraging of the genius of Laurent Schwartz.  The world needs this done.

Stanford, would you be so kind as to put some effort and do this?

So I believe that in the long run, both distribution theory and wavelet theory will represent {\em one discipline} that can give us canonical tools to handle various classes of functions and distributions with clean approaches to computation that will seal up this area for future.  Then Computer Science, Partial Differential Equations, Harmonic Analysis topics can all be taught together so that students of all these disciplines have mastery of basic theory and facilility for use in a wide variety of areas of value to the human race.  

This sort of merging removes the problems where application oriented people are unfamiliar with mathematical issues and mathematically oriented people are not familiar with the computational aspects.  Then this could be distributed as a basic topic across the world with professionalisation.  

You see, Distribution Theory is only esoteric today because we are not yet fully able to concretely leverage them. It's just like trigonometry and calculus which is well-known to billions of students worldwide. This merge is simply natural development of human civilisation, and ought to be made accessible to people of all nations.

\section{Concretisation of Vision}

At Princeton freshman Analysis is from Rudin, and then there are courses on Fourier analysis from Dym-McKean and there are also courses on Partial Differential Equations for Mathematics.

What I see happening in the long run is some course after advanced calculus or Analysis at the baby Rudin level that merges basic aspects of distribution theory and wavelet analysis with concrete algorithms with software tools to understand aspects in computer too.  

You see, I love Dym-McKean's book and Fourier Analysis of Korner and others.  And I have a lot of experience with wavelet analysis.  I think the whole idea of how young people learn the issues of analysis, fourier theory, partial differential equations could benefit from making thing {\em concrete} as well.  They could form a year-long period of sharpening both algorithmic and mathematical aspects at the level right after one masters baby Rudin level Analysis. 

It will be good for the world.  People around the world with degrees will have concrete facility with thought and skills for use in millions of applications.  It will, I think, truly make people more versatile and give a quantum leap to human sophistication in civilised life.

I literally imagine hundreds of millions of young people able to do some of this going into the world and changing the world in a positive way with firmer confidence and better skills.

\section{Wavelet-Distributional Analysis Has Some Developments}

There is some nice work by Katerina Saneva and Jasson Vindas \cite{SV}.  Stanford, why don't you organise a conference to engage people with the larger goals that can be accomplished by the merging of the theory and applications for significant unity of wavelet analysis and distributions for standardisation in analysis?  It would take some organised effort to turn the research work into standard courses and also assisting undergraduate programs to produce the merge.

I can assure you that the merging of distribution theory and wavelet analysis will {\em significantly} transform analysis training and reach and power of the ideas with concreteness on millions of problems of interest around the world.  It is {\em undergraduate and graduate education} in analysis that must change globally.

\section{My Philosophical View}

Years ago I began to think about theoretical physics in around 2008.  In the end, I succeeded in producing a coherent theoretical physics that resolved David Hilbert's sixth problem in the general version of a completely coherent theoretical physics a decade later.  This is the Four-Sphere Theory.  I was able to calculate the radius of all of existence to $R=3075.69$ Mpc, and explain quantum phenomena as a consequence of global geometry of the universe.  Centuries from now, and perhaps already, it will simple and obvious, all elements of Four-Sphere Theory, but it was very difficult to work on this because no one wanted to assist with funding and no one believed that it would succeed.

What you will find, when you examine my work on Four-Sphere Theory, is a deep faith that nature operates by {\em beautiful} mathematical laws.  For example, my S4 Electromagnetic law, which significantly improves Man's understanding of Nature over Schroedinger and Maxwell laws, is a {\em classical wave} equation on spinor fields of a four-sphere of fixed radius.  And that is much better science than relativity and quantum field theory.  

These things lead me to have great faith that beauty does matter in truth, that John Keats' exclamation that "Beauty is Truth and Truth Beauty" was wiser than people have understood since the Romantic Era.

As a result of my successes, I am rather unenthusiastic about the division between pure and applied mathematics.  What we are interested, we humans, are to understand deeply some things that we call mathematical, and understand the thing that we call existence and begin to sense how it is that the two always merge more beautifully.  The beauty is never in purity of pure mathematics; neither is it in the messy complexity of hodge podge of ideas applied gracelessly for pragmatism.  It is in the remarkable miracle when there is an exact correspondence.  Those moments are rare obviously, but this is what the enterprise is really about.  No truly deep mathematics had ever occurred that did not in the end touch some part of existence. 

In any case, wavelet analysis merging with distribution theory from analysis is an opportunity for deeper beauty to distill, that can become part of the set of things that becomes part of our natural habit analytically.  The insight is that it is not the technical details that matter, but understanding in the ways that we model nature in concrete ways that lead to their concrete use with ease by massive numbers of people in massive aspects of the world.

In several centuries, the intuition of functions and their ability to produce models of the world will be part of the intellectual and practical toolset of ordinary high school children and that is already clear.  

We have to {\em plan} these changes and push them forward.  The arena is beautiful and it does not have to stay too difficult for younger people.  It is our duty to grant our beloved people the fruits of the achievements of these realms.  They are deserving of it. 



\section{Stanford Analysis Qual Spring 2012 I.2}

(a) Show that $|x|^{-(n-\beta)} \in \mathcal{S}'(\mathbf{R}^n)$ for $0<\beta<n$.

(b) Show that $f(x) = e^x\cos(e^x)$ lies in $\mathcal{S}'(\mathbf{R})$.

(c) Prove that there is a locally $L^1$ function that does not lie in $\mathcal{S}'$.

What does it mean for a function $f$ to lie in $\mathcal{S}'(\mathbf{R})$?  The meaning is that the functional
\[
j(f)( \phi ) \le \int f(x) \phi(x) dx
\]
is a bounded linear functional on $\mathcal{S}(\mathbf{R}^n)$.  

Fine, let's take a look at $j(f)$ where $f(x) = |x|^{-(n-\beta)}$.  Now Schwartz space is defined elements $\phi$ satisfy
\[
\lim_{|x|\rightarrow \infty} (1+|x|)^p |D^{\alpha} \phi| = 0
\]
for every $p=0,1,\dots$.  So what we want to pull out a supremum from the functional.

\[
j(f)(\phi) \le \int f(x) dx \| (1+|x|)^p \phi \|_{\infty}
\]

Let's take a look at when 
\[
\int f(x) |x|^r dx < \infty
\]
for any $r$.  We don't like the singularity at zero too much so we just consider $\epsilon>0$ small and break up $\mathbf{R}^n= B(\epsilon) \cup A$.

Now
\[
\int_{B(\epsilon)} f(x) |x|^r dx = C V(\epsilon) \int_0^{\epsilon} s^{-(n-\beta)+r} ds
\]
using polar coordinates and
\[
\int_{\epsilon}^\infty s^{-(n-\beta)+r} ds
\]
The second integral is finite whenever $n-\beta+r< -1$, i.e. $n < \beta - 1 - r$.  The first integral is finite whenever $r > n-\beta$.  

For (b) we want to find some $r$ such that
\[
\int e^x \cos(e^x) x^{-r} dx <\infty
\]

For (c) we have to find some locally $L^1$ function such that 
\[
\int f(x) x^{-r} dx 
\]
diverges for all $r$.

\begin{thebibliography}{CCCC}
\bibitem{SV}{Katerina Saneva and Jasson Vindas, Wavelet expansions and asymptotic behavior of distributions, J. Math. Anal. Appl. 370 (2010) 543–554}
\end{thebibliography}

\end{document}