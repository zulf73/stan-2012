\documentclass{amsart}
\usepackage{graphicx}
\graphicspath{{./}}
\usepackage{hyperref}
\usepackage{csvsimple}
\usepackage{longtable}
\usepackage{lscape}
\usepackage{epigraph}
\title{Zulf's Stanford Analysis 2012 Qual Problem 4}
\author{Zulfikar Moinuddin Ahmed}
\date{\today}
\begin{document}
\maketitle

\newcommand{\zbar}{\bar{z}}

\newcommand{\ddzbar}[1]{\ensuremath{\frac{\partial{#1}}{\partial\bar{z}}}}

\newcommand{\ddz}[1]{\ensuremath{\frac{\partial{#1}}{\partial z}}}

\section{Problem 4}

Suppose $f,g$ are holomorphic in open set $\\Omega \subset \mathbf{C}$ and $|f|^2 + |g|^2$ is constant.  Show $f$ and $g$ are both constant.

\section{The Derivative Approach}

If $f$ is holomorphic, we have 
\[
\ddzbar{f} = 0
\]

What we want to do is just take partial derivatives in $z$ and $\bar{z}$ of $|f|^2 + |g|^2$ and set it equal to zero and then attempt to deduce that 
\[
\ddz{f} = 0
\]
and
\[
\ddz{g} = 0
\]
In order to see if this approach leads to something without a messy calculation, let's just try to prove a simpler statement.  We want to prove if $|f|^2$ is constant then $f$ is constant when $f$ is holomorphic.

\begin{align*}
\ddzbar{|f|^2} &= \ddz{ f\bar{f}}\\
 &= \ddz{f} \bar{f} + f \ddz{\bar{f}}\\
 &= f \ddz{\bar{f}}.
\end{align*}
This is from a conjugate magic. Then we assume $f\not= 0$ since otherwise we have constant $f=0$, and then divide 
\[
0 = \ddz{f} f
\]
by $f$ and we have in addition to 
\[
\ddzbar{f}=0
\]
the derivative in the orthogonal $\zbar$ direction also zero and can conclude $f$ is constant.

Now let's do the same for $|f|^2 + |g|^2$. Taking $\zbar$ derivatives here,
\[
\ddzbar{(|f|^2+|g|^2)} = f \ddzbar{\bar{f}} + g \ddzbar{\bar{g}}
\]
Now let's see.  If either $f$ or $g$ is identically zero, our previous argument gives the result, so let's assume $f,g\not=0$.  

We use
\[
\bar{f}\ddz{f} + \bar{g}\ddz{g} = 0
\]
in the region $\Omega$.  We can get $bar{g}/\bar{f}$ is holomorphic from this.  
\end{document}

