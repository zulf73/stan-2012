\documentclass{amsart}
\usepackage{graphicx}
\graphicspath{{./}}
\usepackage{hyperref}
\usepackage{csvsimple}
\usepackage{longtable}
\usepackage{lscape}
\usepackage{epigraph}
\title{Zulf Discovers Large Life Satisfaction in Standard of Living Compared to Parents} 
\author{Zulfikar Moinuddin Ahmed}
\date{\today}
\begin{document}
\maketitle

\section{The Numbers}

\begin{align*}
P(S|\mathrm{Hi}) &= 0.421\\
P(S|\mathrm{Lo}) &= 0.158
\end{align*}

\section{Life Satisfaction is a Mysterious Variable}

Life Satisfaction and its precise sensitivities are totally mysterious still.  This is one of the cases where there is just voluminous opinions and ideologies and all sorts of confused thought and no clarity.  We are still in pre-scientific era with regards life satisfaction.  Progress will occur when we have precise quantitative scientific models that are precise.  

The general idea that standard of living relative to parents might have an effect on life satisfaction is reasonable.  And yet, and in this case there is some quantitative support so it is a strong candidate.  But we do not yet have a precise accurate predictive model for life satisfaction yet.


\section{Life Satisfaction Higher For Those Who Have Gone Without Shelter}

This is a total surprise.

Precise numbers are here:
\begin{align*}
P(H|\mathrm{Shel}) &= 0.877 \\
P(H|\mathrm{NoShel}) &= 0.762 \\
P( S | \mathrm{Shel} ) &= 0.20\\
P( S | \mathrm{NoShel} ) = 0.377
\end{align*}

I double-checked.  This is in fact the case.  People who have gone without shelter in the past 12 months actually have much higher Life Satisfaction than those without.
\end{document}
